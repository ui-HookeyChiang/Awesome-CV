%-------------------------------------------------------------------------------
%	SECTION TITLE
%-------------------------------------------------------------------------------
\cvsection{Work Experience}


%-------------------------------------------------------------------------------
%	CONTENT
%-------------------------------------------------------------------------------
\begin{cventries}

%---------------------------------------------------------
  \cventry
    {UniFi-OS Storage Software Engineer} % Job title
    {Ubiquiti Inc.} % Organization
    {Taiwan/Remote} % Location
    {Mar. 2022 - Present} % Date(s)
    {
      \begin{cvitems} % Description(s) of tasks/responsibilities
        \item Dedicated to UNAS, including NPI deliveries, test plans, and
          feature developements on Debian.
        \begin{itemize}
          \item Researched and developed cohesive user-friendly NAS solutions
            involving network discovery, Samba, snapshots, data backup, data
            integrity, system-data migration; shared product visions,
            specifications, documents, code snippets in Confluence with
            cross-functional teams.
          \item Upgraded Linux from 4.19 to 5.10, encountering code
            incompatibilities and regressions rooted in PCI, RAID, PHY, AHCI,
            resolved by cautiously backtracking mainline and SDK. Holistically
            verified system stability and performance with stress-ng, fio,
            xfstest, packetdriller, and Phoronix Test Suite. Moreover,
            unleashed IO with the mq-deadline scheduler, Btrfs xxhash, and the
            sendfile syscall.
          \item Reduced cross-device copies, quota accounting and data leakage
            with the FS layout redesign.
          \item Worked around the issue of hours-long unresponsiveness caused
            by Btrfs subvolume deletions.
          \item Tuned Btrfs configs on atime, space\_cache, quota usage to
            achieve better 8\% read and 5\% write.
        \end{itemize}
        \item Brought system stability and performance improvements to all
          UniFi-OS products.
        \begin{itemize}
          \item Reduced overhead of a Python CLI by 65\% usage in CPU, 50\% in
            memory, causing 15\%-to-50\% less in system load average.
            Introduced a gRPC state exporter for lightweight streaming
            notifications.
          \item Fixed an OOM bug due to misbehaving socket forward allocation on
            a 64kB page-sized system.
        \end{itemize}
      \end{cvitems}
    }

%---------------------------------------------------------
  \cventry
    {Software Engineer} % Job title
    {QNAP Systems Inc.} % Organization
    {Taiwan} % Location
    {Jan. 2018 - Feb. 2022} % Date(s)
    {
      \begin{cvitems} % Description(s) of tasks/responsibilities
        \item {Took full part in Hybridmount, a FS as cloud storage
          gateway, deployed on 190K QNAP devices as of 2021, supporting video
          streaming and accessible via common protocols, e.g., SMB, NFS, and FTP.}
        \begin{itemize}
          \item {Designed the database schema, CLI, IPC protocol, and loopback-fs cache
            management.}
          \item {Maintained software quality by composing unit, integration, and
              stress tests, incorporating public FS test suites (pjdfstest) and
            benchmarks (filebench), and building automated CI pipelines using
            Jenkins and Docker. Handled fewer than 20 support tickets since the
            first official release.}
          \item {Reduced duration of compiling codes from 30 minutes to fewer
            than 10 minutes.}
          \item {Reduced the time to create 50 million files from 60 days
            on SSDs to less than 7 days on HDDs.}
          \item {Continuously improved runtime performance, achieving a 50\%
            increase in sequential write speed, 10\% random read, 300\% metadata
            operations through various optimizations, including filename
            hashing, file read-ahead, multi-threading, fine-grained locking, VFS
            caching, database index.}
          \item {Successfully utilized VFS dcache by carefully handling unicode
            normalization and case sensitivity.}
        \end{itemize}
        \item {Implemented features in deduplication filesystem, write-back
            cache and chunk defragmentation. Refactored read/write operations,
            reducing the related codebase by 50\% and additionally resolving
          abnormal data display at chunk edges.}
        \item {Analyzed and created POC of a POSIX-compliant cloud-tiering file
          system with FUSE, IPFS in golang.}
      \end{cvitems}
    }

%---------------------------------------------------------
\end{cventries}
